\clearpage
\section{User Interface}\label{sec:ui}
We have briefly covered the user interface in Section~\ref{sec:intro}. Here we provide more instructions on interacting with each window. We will also explain keyboard shortcuts and mouse operations in this section.

\subsection{Copter Property Window}
This window displays mass and moment of inertia information:

\begin{itemize}
  \item \textbf{Mass} Computing mass is straightforward as it only requires summing up the mass of each part. Technically, each component is defined by a closed surface mesh and a density value. We will assume this mesh is solid and use the divergence theorem to analytically compute its mass. 

  \item \textbf{Moment of Inertia} The moment of inertia is a $3\times3$ symmetric positive semidefinite matrix, defined on a local frame $\mathcal{F}$ rigidly attached to the copter. $\mathcal{F}$ is computed and fixed once an initial design is loaded. The origin of $\mathcal{F}$ is the center of mass and its three axes are parallel to the world frame axes. Recall that the world frame we use is north-east-down, where $x$, $y$ and $z$ point to the north (red), east (green) and down (not shown in the window as it is below the ground) direction.
\end{itemize}

The moment of inertia determines the torque need to rotate the copter. It is generally recommended that you place your initial design in a way that its heading is along the $x$ axis with zero roll angle. In this way the diagonal elements in the matrix reveal the resistance of the copter to rolling, pitching, and yawing respectively.

\subsection{IMU Status Window}
This window shows the reading from a virtual sensor attached to the copter, and the ground truth status. Both consist of the following information:

\begin{itemize}
  \item \textbf{Position} These three numbers show where the center of mass is located in the world frame. As $z$ axis points down to the ground, positive altitude corresponds to a negative $z$ value.
  
  \item \textbf{Euler Angle} These are the roll, pitch and yaw angles. Initially the body axes are parallel to world axes. We first rotate it around $z$ by yaw, then rotate along the new $y$ axis of the body frame by pitch, then finally rotate around the new $x$ axis of the body frame by roll.

  \item \textbf{Velocity} The time derivative of position.

  \item \textbf{Euler Rate} The time derivative of Euler angles.
\end{itemize}

\subsection{Propeller Data Window}
This window contains information about rotors, with one sub-window for each. The rotors are ordered by their occurrence in the file. Each sub-window contains:

\begin{itemize}
  \item \textbf{Position} This shows where the motor is located in the \textbf{body} frame. Their values get updated during the design process and fixed in simulation.

  \item \textbf{Direction} This unit vector shows where this motor points to in the \textbf{body} frame. Typically it is close to $(0,0,-1)$.

  \item \textbf{Spinning} A string shown here indicates whether the propeller spins clockwisely (CW) or counter-clockwisely (CCW), defined in a view where the motor is pointing towards you.

  \item \textbf{Speed} This shows the rotation speed of the propeller, in RPM (round per minute).

  \item \textbf{Thrust} This is the magnitude of the thrust generated by the propeller.
\end{itemize}
Both speed and thrust are calculated from fitting measurement data provided in the file. Thrust is also clamped between $0$ and the measured maximum.

\subsection{Battery Status Window}
This progress bar works as an indicator of the battery life. During a flight we collect the thrust of each rotor, and based on the measurement data provided in the file, we estimate the current sent to each rotor. Dividing the remaining capacity by the current gives a rough estimation of time left before the battery becomes completely drained. The progress bar will turn yellow when it is $60\%$, and become red when below $20\%$.

\subsection{Design Panel Window}
This is the main window people will be working on when designing the copter. It consists of a sub-window that allows you to directly change full parameters, another to manipulate equivalent reduced parameters, and a simulation panel.

\subsubsection{Full Parameters}
As we will see in Section~\ref{sec:file}, parts like tubes, connectors and motors are parametrized by their geometric properties. This window displays all parameters labeled by their names, with an extension of indices. Although you are allowed to slide each single value, note that these parameters are constrained by the way you build the copter, so changing one parameter may affect the others as well.

Technically, let $\mathbf{x}$ be a vector of all parameters listed in the ``Full Parameter'' window. We build all constraints as linear functions of $\mathbf{x}$, which are compactly represented by $\mathbf{Ax}=\mathbf{b}$ and $\mathbf{Cx}\leq\mathbf{d}$. When you slide a single parameter in the window, we store all parameter values currently shown in the window into a vector $\mathbf{x}_0$, and the new $\mathbf{x}$ is determined by solving the optimization problem below:
\begin{equation}
\begin{aligned}
\min_{\mathbf{x}}&\quad \|\mathbf{x} - \mathbf{x}_0\|\\
s.t.&\quad \mathbf{Ax}=\mathbf{b}\\
&\quad \mathbf{Cx}\leq\mathbf{d}
\end{aligned}
\end{equation}
In other words, we find in the feasible set the closest solution to the parameter values indicated by users from the window.

\subsubsection{Reduced Parameters}
It is common that tens or hundreds of parameters are introduced in a design while most of them are limited by equality constraints. In this case using reduced parameters can significantly reduce the design space. Mathematically, the solution of $\mathbf{Ax}=\mathbf{b}$ can be fully characterized by vectors in its null space. Let $\mathbf{Ax}_0=\mathbf{b}$, and write $\mathbf{x}=\mathbf{x}_0+\mathbf{Ey}$ where $\mathbf{E}$ spans the null space of $\mathbf{A}$. Now the equality constraints are removed and we can focus on a much lower dimensional vector $\mathbf{y}$.

The reduced parametric representation brings its own drawbacks though. Perhaps the biggest issue here is that $\mathbf{y}$ does not have a straightforward physical meaning any more. Each element does not correspond to a single component, but might affect the global design in an unintuitive way.

Unlike the full parameters, it is easier to update the reduced parameters when they get updated in the window. Let $\mathbf{Fy}\leq{g}$ be the only inequality constraints on $\mathbf{y}$. When one $y_i$ is updated, we fix other $y_i$s and calculate the feasible range of $y_i$ and reflect it in the window. If some $y_i$ freezes on the boundary of $\mathbf{Fy}\leq\mathbf{g}$, its background then becomes red to imply this $y_i$ is not changeable.

\subsubsection{Simulation}
This window contains a button to trigger the simulation process. Your design is then set to track the position and heading of the blue arrow in the scene, which can be changed by keyboard shortcuts discussed below. All design parameters are frozen during flight unless you press the button again and switch back to the design process.

Behind this simulation button we execute a $30$Hz loop to control the flight. Within each iteration we do the following things in order:
\begin{itemize}
  \item Collect the sensor data and target information.

  \item Execute the control policy based on them.

  \item Send control signals to each rotor.

  \item Use thrust and torque from each rotor to update the motion.
\end{itemize}
Note that depending on your graphics card the actual FPS (frame per second) may be above or below 30. So the time flow you feel in the window is not exactly the actual motion you expect to see in your real copter.

\subsection{Keyboard Shortcuts}
The following keys are used as shortcuts. Most are activated only in simulation.
\begin{itemize}
  \item \textbf{Left/Right/Up/Down Arrow} This four keys control the horizontal position of the target (blue arrow in the window). The default controller will try to drag the copter to track the blue arrow as close as possible.

  \item \textbf{`w' and `s'} These two keys adjust the vertical location of the blue arrow, which in turn controls the altitude of the copter.

  \item \textbf{`a' and `d'} These keys rotate the blue arrow on the horizontal plane, which changes the desired heading of the copter in simulation.

  \item \textbf{`p'} When pressing `p' you can pause/resume the simulation.
\end{itemize}

\subsection{Mouse Operations}
Mouse gestures are used to control the camera view. Specifically, the following operations are supported:
\begin{itemize}
  \item \textbf{Zooming in/out} By scrolling the mouse wheel you can zoom the camera to get closer/farther to the scene.

  \item \textbf{Rotation} You can press the mouse wheel and drag to rotate the camera.

  \item \textbf{Panning} Press shift and the mouse wheel and drag your mouse at the same time to change the camera location.
\end{itemize}